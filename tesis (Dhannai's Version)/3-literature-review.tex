\documentclass[tesis.tex]{subfiles}

\begin{document}
    
\section{Literature review}

\subsection{Bar-built estuaries in the ecosystem and the community}

Climate change is affecting multiple marine ecosystems globally \citep{hewitt2016multiple}. Its been detected that the global oceanic oxygen content has decreased during the last five decades \citep{schmidtko2017decline} and that air temperature is increasing in oceans \citep{omstedt2004baltic, jones1999surface}. Also, some studies expect that the absolute mean sea level on Chilean coasts rises between 0.35 to 0.74 m in the next 80 years \citep{winckler2020evidence}. The effects of climate change can put at risk the coastal zones, including estuaries and coastal lagoons which are especially abundant ecosystems in flora and fauna.\\

In addition, there is evidence that there is a decrease in surface wind speeds in Northern Europe \citep{woolway2017atmospheric} and an increase in along-shore winds in the Chilean coastal zone \citep{winckler2020evidence}. It is known that changes in surface wind speed affect the number of days that a lake is stratified, which affects the nutrient availability and quality of a waterbody, changing the amount of oxygen present in deep waters \citep{woolway2017atmospheric}. It is important to study wind effects in estuaries to be able to quantify how wind-speed changes will affect these environments.\\

In central Chile, there is a decrease in river discharges affecting buoyancy and stratification \citep{winckler2020evidence}, which can be causing a wide range of changes in estuarine and marine ecosystems, including changes in oxygen availability. These changes can impact fish populations and other autotrophic organisms.\\

The importance of intermittently closed estuaries goes beyond local impacts. These estuaries can accumulate sediment and minerals while the inlet is closed \citep{thorne2021wetlands}, and in rainy seasons they open their mouth naturally because of the increase in freshwater inflow \citep{hoeksema2018factors}. This process settles sediments to the nearby marshes helping to maintain their elevation according to the sea level, mitigating the consequences of sea level rise \citep{thorne2021wetlands}. Usually, the mouth is exposed to artificial openings to avoid flooding the surrounding lands \citep{Behrens2013}, which does not allow the sediments to set in the marsh platform \citep{thorne2021wetlands}, not allowing them to keep their normal elevation that protects the coastal zone from the sea level.\\

Rising river discharge due to climate change could lead to increase erosion and the number of suspended particles of sediment in the water \citep{whitfield1994changes}. Enhanced sediment concentration could lead to accumulation in the estuary making the inlet close, changing the equilibrium of opened and closed state of the sand bar, which along with the increase of freshwater input could flood the surrounding land \citep{peeters2009currents}. Consequently, depending on the vegetation present and its oxygen demand, deep-water oxygen may be reduced or suppressed \citep{Kelly2018, Largier2021}. Also, the density of the surface waters will be reduced and thus could change the estuary behavior to external factors such as wind stress. \\

Bar-built estuaries are under continuous anthropogenic stress due to their closeness to human settlements \citep{clark2019systematic} and their productive importance. Dams constructed upstream for water storage reduce the freshwater that goes to the ocean, causing the retention of suspended sediments. This results in a change in the morphology of the estuary due to not receiving the sediments that used to accumulate in the inlet, leading to premature scour of the sand bar \citep{peeters2009currents}. Also, to prevent the flood of roads or agricultural lands that settle nearby, the community plan the opening of the inlet artificially, which could result on abrupt changes on the estuary ecosystem \cite{Behrens2013}. \\

\subsection{How bar-built estuaries are studied in Chile and around the world}

\cite{mcsweeney2017intermittently} studied the bar-built estuaries all around the world and their climatic, marine, and fluvial conditions to classify them and quantify the drivers of their distribution in each continent. That can "allow predictions of estuary response to climate change and human impacts to be made and to ultimately assist with integrated coastal management into the future".\\

\cite{dussaillant2009} studied a Chilean coastal lagoon in its open and closed state and observed that in its closed state the rainfall influence was not important except for the storms that open the inlet to the sea. He also observed that wind effects can be important during the disconnected phase to the ocean, producing a slope in the lagoon levels. He studied the connected phase using a general pattern, spectral, and Fourier analysis.\\

In their study, \cite{Gale2006} investigated the dynamics of Intermittently Closed and Open Lakes and Lagoons (ICOLLs) during their closed state. The research findings revealed that the presence of stratification ($N > 0.1 s^{-1}$) can lead to  depletion of oxygen in the bottom waters. This particular factor has been associated with fish kills in Pescadero, as documented by \citep{largier2015}.\\

\cite{Kelly2018} conducted a study on Lough Furnace, a naturally deep saline lagoon characterized by restricted tidal inflows and long periods of low freshwater input, which showed continuous vertical stratification and anoxia in the lower layers. The study found that specific tidal events and wind-driven upwelling could oxygenate the deeper layers, revealing a correlation between tidal influence and wind stress in vertical mixing.\\

\cite{Behrens2016} observed the salt intrusion in a bar-built estuary and its differences between closed and open state conditions. The study found the presence of alternating shallow sills and deep pools, which act to trap the salt after intrusion, and suggested that internal seiche motions in the outer estuary initiate the intrusion by lifting saline water in the pycnocline high enough to crest the sills. This salinity intrusion extends to distances of several kilometers from the beach when the estuary is in closed state.\\

In Rodeo Lagoon, a shallow strongly-stratified lagoon, \citep{Cousins2010} investigated the effect of stratification on water column parameters such as salinity, dissolved oxygen, and nutrient levels. They found that stratification causes a significant suppression of turbulence below the pycnocline, resulting in the confinement of nutrients in the lower layer for several months. The study revealed that wind is the primary source of mixing in the lagoon, and destratification by wind allows for the redistribution of nutrients from the bottom brackish layer.\\

\subsection{Hydrodynamics of a stratified waterbody}

In nature, stratified waterbodies can be found not only in estuaries \citep{human2016} but also in lakes \citep{Valerio2012, Imam2013, Coman2012} or coastal lagoons \citep{Cousins2010}. Although lakes are usually studied as thermally stratified water systems, they exhibit comparable hydrodynamics to thermal-haline stratified coastal waterbodies. In estuaries, when the tidal connection with the ocean is limited, water circulation is driven by wind and freshwater inflow, resulting in similar dynamics to lakes in a smaller scale. \\

In stratified lakes or estuaries, it is common to find a two-layered system with the presence of an interface of finite thickness, which is a third middle layer. This middle layer can be observed as a gradient of density or temperature that separates the upper layer from the lower layer. The interface layer thickness is an important parameter that can impact the dynamics of the water column in these types of waterbodies \citep{simpson1974fronts}.\\

Depending on the strength and duration of wind forcing, the lake or estuary can manifest an upwelling response. Upwelling is a significant process that occurs in stratified estuaries and lakes, influencing their hydrodynamics and ecosystem dynamics. It involves the vertical transport of nutrient-rich, deep oceanic waters to the surface, promoting enhanced biological productivity and supporting diverse marine ecosystems \citep{gupta2022nutrient}. One parameter commonly used to characterize upwelling in estuaries is the Wedderburn number \citep{Imberger1982}. By quantifying the Wedderburn number, researchers can assess the strength and effectiveness of upwelling processes in bar-built estuaries, aiding in the understanding and management of these dynamic coastal environments.\\

The wind's energy is the primary source of energy for the water column's circulation, and it can cause an upwelling response when it is strong enough to overcome the stratification of the water layers. Upwelling occurs when the wind's energy forces the lower layer of water to move upward, bringing nutrients and other materials to the surface that can stimulate primary productivity in the water column \citep{macintyre2010ecosystem}. \\

\cite{roberts2021setup} studied the setup and relaxation of spring upwelling in a deep, rotationally influenced lake, Lake Tahoe, using a combination of field observations and numerical modeling to investigate the mechanisms that cause the upwelling of deep water in the lake. They found that the setup of upwelling was caused by the wind-induced mixing of the upper layer of the lake, which resulted in a deeper mixed layer and the buildup of potential energy. The relaxation of upwelling occurred when the wind stopped, and the potential energy was converted into kinetic energy, which led to the downwelling of surface water. These findings provide new insights into the mechanisms that control the dynamics of upwelling in deep lakes and could help inform the management of these ecosystems.\\

The Wedderburn number was design for rectangular basins, but this approach is not too close to reality, where basins can be of multiple and irregular shapes. \cite{Shintani2010} used a numerical model to demonstrate that the upwelling of deep water in lakes with any geometry can be described using the Wedderburn number as a function of the Richardson number, the buoyancy frequency, and the Rossby number. This Wedderburn number is not a detailed estimate of the interface behavior, therefore provides a scale for the seiching. These results provide a better understanding of the physical processes that drive the upwelling of deep water in lakes and could help improve the management of these ecosystems.\\

Wind stress also induces a tilting of at the interface between layers of different densities. This interface tilt results in the transfer of momentum from the wind to the water column, leading to mixing and vertical exchange processes. The tilt can be quantified with the internal wave period $T_1$, and also we can consider that after commencement of the wind stress, it takes $T_1/4$ for the condition identified by W to occur.\\

\cite{Monismith1985} discussed that a three-layered fluid has a similar behavior as a two-layered fluid when the upper layer is shallow. This is because the shallow upper layer behaves like a mixed layer, while the middle layer acts as an interface layer separating the mixed layer from the lower layer. When the upper layer accelerates due to a wind forcing in the surface, the mixed layer starts to deepen rapidly, while the upper layer tilts and might upwell \citep{monismith2006vertical}.\\

The response of stratified lakes or estuaries to wind forcing events can be complex, involving interactions between the layers of the water column, upwelling responses, and changes in the water column's stability \citep{jayaweera2019turbulent}. Upwelling occurs when the wind's energy forces the lower layer of water to move upward, bringing nutrients and other materials to the surface that can stimulate primary productivity in the water column \citep{bastidas2021comparison}. The thickness of the interface layer is an important parameter that can impact the dynamics of the water column in these types of waterbodies \citep{xu2017vertical}. Factors such as wind strength and duration, water temperature, and the presence of nutrient-rich layers in the water column can all affect the response of stratified lakes or estuaries to wind forcing events \citep{nidheesh2018stratification}.\\

\subsection{Pescadero estuary studies}

Recent studies on the Pescadero Estuary have focused on fish kills that occur when the sandbar closes, leading to the creation of an anaerobic environment in the bottom waters \citep{sloan2006ecological}. Geochemical analysis of sediments has shown that the transition from the closed to open state leads to poor water conditions within the Pescadero Estuary, with many indicators reaching values outside the range of optimal conditions for fish or aquatic life \citep{richards2018}. \cite{huber2020environmental} documented one of the breach-induced fish kills events, demonstrating a case of ecosystem function loss caused by chronic degradation of water quality during the closed estuarine state. While previous literature on the Pescadero Estuary has focused on management plans for biological productivity \citep{curry1985pescadero} or preserving the hydrology of n the estuary \citep{williams1990pescadero}, these recent studies highlight the importance of understanding the physical and chemical dynamics of bar-built estuaries.\\

In addition to the effects of sandbar closure, physical phenomena such as the effects of the constriction generated by the mouth in its open state have been studied. \cite{williams2016} observed that wave setup and tides set the estuarine water level, while the mouth sandbar limits ocean gravity waves from entering the estuary but permits infragravity motions to pass through the inlet, inducing energetically important high velocities. These studies highlight the strong dependence of hydrodynamics of small bar-built estuaries on nearshore processes and the need to understand the complex interactions between external factors and estuary dynamics.\\

\end{document}