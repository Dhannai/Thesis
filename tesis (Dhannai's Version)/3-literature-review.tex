\documentclass[tesis.tex]{subfiles}

\begin{document}
    
\section{Literature review}

\subsection{Bar-built estuaries in the ecosystem and the community}

Climate change is affecting multiple marine ecosystems globally \citep{hewitt2016multiple}. Its been detected that the global oceanic oxygen content has decreased during the last five decades \citep{schmidtko2017decline} and that air temperature is increasing in oceans \citep{omstedt2004baltic, jones1999surface} which according to models can affect stratification in northwest European continental shelf and Baltic Sea due to a decrease of salinity at the surface \citep{hordoir2012effect, holt2010potential} changing the number of days that stratification is present causing impact in nutrient flux. Also, some studies expect that the absolute mean sea level on Chilean coasts rises between 0.35 to 0.74 m in the next 80 years \citep{winckler2020evidence}. The effects of climate change can put at risk the coastal zones, including estuaries and coastal lagoons which are especially abundant ecosystems in flora and fauna.\\

In addition, there is evidence that there is a decrease in surface wind speeds in Northern Europe \citep{woolway2017atmospheric} and an increase in along-shore winds in the Chilean coastal zone \citep{winckler2020evidence}. It is known that changes in surface wind speed affect the number of days that a lake is stratified, which affects the nutrient availability and quality of a waterbody, changing the amount of oxygen present in deep waters \citep{woolway2017atmospheric}. It is important to study wind effects in estuaries to be able to quantify how wind-speed changes will affect these environments.\\

In central Chile, there is a decrease in river discharges affecting buoyancy and stratification \citep{winckler2020evidence}, which can be causing a wide range of changes in estuarine and marine ecosystems, including changes in oxygen availability. These changes can impact fish populations and other autotrophic organisms.\\

The importance of intermittently closed estuaries goes beyond local impacts. These estuaries can accumulate sediment and minerals while the inlet is closed \citep{thorne2021wetlands}, and in rainy seasons they open the mouth naturally because of the increase in freshwater inflow \citep{hoeksema2018factors}. This process settles sediments to the near marshes helping to maintain their elevation according to the sea level, mitigating the consequences of sea level rise \citep{thorne2021wetlands}. On the other hand, it is very common opening the mouth artificially to avoid flooding the near lands \citep{Behrens2013} which doesn't allow the sediments to set correctly in the marsh platform  \citep{thorne2021wetlands}. ENSO (El Niño Southern Oscillation) is the principal cause of the opening and closure of the mouth \citep{mcsweeney2017intermittently}, but this phenomenon can change its occurrence in the next years, affecting estuaries' dynamics and water quality all around the world \citep{thorne2021wetlands}.\\

Climate change is affecting bar-built estuaries' dynamics and water quality. Increasing river discharge due to more precipitation could lead to increase erosion and the number of suspended particles of sediment in the water. Enhanced sediment concentration could lead to accumulation in the estuary making the inlet close, changing the equilibrium of opened and closed state of the sand bar, which along with the increase of freshwater input could flood the surrounding land \citep{peeters2009currents}. Consequently, depending on the vegetation present and its oxygen demand, deep-water oxygen may be reduced or suppressed \citep{Kelly2018, Largier2021}. Also, the density of the surface waters will be reduced and thus could change the estuary behavior to external factors such as wind stress. \\

On the other hand, bar-built estuaries are under continuous anthropogenic stress due to their closeness to human settlements \citep{clark2019systematic} and their productive importance. Dams constructed upstream for water storage reduce the freshwater that goes to the ocean, causing the retention of suspended sediments. This results in a change in the morphology of the estuary due to not receiving the sediments that used to accumulate in the inlet, leading to premature scour of the sand bar \citep{peeters2009currents}. Also, to prevent the flood of roads or agricultural lands that settle nearby, the community plan the opening of the inlet artificially, which could result on abrupt changes on the estuary ecosystem \cite{Behrens2013}. \\

\subsection{How bar-built estuaries are studied in Chile and around the world}

There are plenty of methods and instrumental techniques to measure the behavior of estuaries and lakes at a small scale \citep{Wuest2003}, methods that can be used with new data and get improved for future works and be more specific for the different types of waterbodies. \cite{mcsweeney2017intermittently} studied the bar-built estuaries all around the world and their climatic, marine, and fluvial conditions to classify them and quantify the drivers of their distribution in each continent. That can "allow predictions of estuary response to climate change and human impacts to be made and to ultimately assist with integrated coastal management into the future".\\

\cite{dussaillant2009} studied a Chilean coastal lagoon in its open and closed state and observed that in its closed state the rainfall influence was not important except for the storms that open the inlet to the sea. He also observed that wind is very important in water level fluctuations in the disconnected phase. He studied the connected phase using a general pattern, spectral, and Fourier analysis.\\

\cite{Gale2006} observed that in stratified waterbodies, when the vertical exchange is limited, oxygen depletion can occur, causing hypoxia and anoxia, a factor that is related to fish kills in Pescadero \citep{largier2015}. \cite{Kelly2018} proposed that tidal influence oxygenated the deeper layers in a saline lagoon in some specific events and observed that the same conditions were present when there was wind-driven upwelling, showing a relation between tidal influence and wind stress in vertical mixing.\\

\cite{Behrens2016} observed the salt intrusion in a bar-built estuary and its differences between closed and open state conditions. The study found the presence of alternating shallow sills and deep pools, which act to trap the salt after intrusion, and suggested that internal seiche motions in the outer estuary initiate the intrusion by lifting saline water in the pycnocline high enough to crest the sills. This salinity intrusion extends to distances of several kilometers from the beach.\\

Studies carried out in Rodeo Lagoon \citep{Cousins2010}, a shallow strongly-stratified lagoon, found that stratification leads to a pronounced suppression of turbulence below the pycnocline and confines nutrients released from the sediment into the lower layer. Bottom water can be confined for several months, compared to the rapidly flushed overlying fresh layer. They observed that in the lagoon wind is the dominant source of mixing because of a lack of other energy inputs and destratification by wind mixing allows for the redistribution of nutrients from the bottom brackish layer.\\

\subsection{Hydrodynamics of a stratified waterbody}

In the nature stratified waterbodies can be found other than estuaries, in lakes or coastal lagoons

\subsection{Pescadero estuary studies}

Pescadero estuary has literature related to management plans focusing on productivity \citep{curry1985pescadero} or in preserve the hydrology of the estuary \citep{williams1990pescadero}. But recent studies have been motivated on the fish kills that have been observed in the last years, signaling that when the sandbar closes stratification leads to the creation of an anaerobic environment in bottom waters \citep{sloan2006ecological}. Also, geochemical analysis to sediments showed that the transition from closed to open state leads to poor water conditions within the Pescadero Estuary, with many indicators reaching values that are outside the range of optimal conditions for fish or aquatic life \citep{richards2018}. \\

In addition, it has been studied more physical phenomena like the effects of the constriction that generates the mouth in its open state, showing a discontinuous tidal forcing in the estuary \citep{williams2016}. \cite{williams2016} observed that wave setup and tides set the estuarine water level, while the mouth sandbar limits ocean gravity waves to enter the estuary but permits infragravity motions to pass through the inlet, which induced energetically important high velocities, highlighting the strong dependence of hydrodynamics of small bar-built estuaries on nearshore processes. Also, hydrodynamic processes in Pescadero are comparable to similar estuaries along the western coast of the Americas as well as in Australia, South Africa, and in estuaries in Mediterranean climates on the Atlantic west coast of Europe, as well as in shallow sandy inlets elsewhere.\\



\end{document}