\documentclass[tesis.tex]{subfiles}

\begin{document}

\section{Objectives}

\subsection{General objective}

The main goal of the present work is to comprehensively understand and quantify the drivers of stratification and mixing in closed bar-built estuaries. These estuaries are characterized by a mouth that is periodically closed by sandbars, leading to unique hydrodynamic conditions. By investigating and analyzing the factors that influence stratification and mixing in such estuarine systems, this research aims to provide insights into the physical processes that govern water column structure and dynamics in these environments. The findings of this study will contribute to a deeper understanding of how external forces, such as wind stress, freshwater input, and wave overtopping, interact with the complex estuarine morphology to influence stratification and mixing patterns. This knowledge will have practical implications for the management and conservation of closed bar-built estuaries, as it can inform decision-making related to water quality, ecosystem health, and coastal resource management. Our case study is the Pescadero Estuary, a bar-built estuary in California that represents many other small inlet systems in mediterranean climates worldwide. \\

\subsection{Specific objectives}

The specific objectives of this study are:
\begin{enumerate}
    \item [(1)]  To investigate the velocity and density variability in a small, highly stratified estuary during its closed state. The findings will contribute to our understanding of water column dynamics in closed bar-built estuaries.
    \item [(2)] To identify the influence of wind stress on the hydrodynamics of a closed bar-built estuary. This research will contribute to a better understanding of the relationship between wind stress and estuarine stratification.
    \item [(3)] To study the wind-estuary interaction and the effects of other external factors such as water inflow and wave overtopping. This is crucial for understanding the dynamics of estuarine stratification and mixing, which have significant implications for water quality, ecosystem health, and resource management.
\end{enumerate}


\end{document}