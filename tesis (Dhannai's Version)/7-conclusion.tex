\documentclass[tesis.tex]{subfiles}
\begin{document}
  
\section{Conclusion}

The analyzes carried out showed how wind, freshwater inflow, and tide are influencing the stratification of the Pescadero estuary. We could notice that the constant discharge from Pescadero and Butano creek is changing water level and stratification in the estuary by increasing the epilimnion thickness. The latter, changes the lagoon response to wind forcing which was proved by buoyancy frequency behavior and potential energy anomaly. Consequently, the vertical exchange was reduced, limiting deep-water renewal. The latter could cause oxygen depletion which is associated with fish kills \citep{Kelly2018}.\\

We could observe that there was saltwater inflow caused by wave overtopping in the estuary, but it was not big enough to change stratification in long term. Anyways there was a slow increase in the deeper layer of DC location that could be driven by wave overtopping. We also observed some increases in the stratification on the other layers but were only temporary and small. In addition, the wave overtopping did not cause an increase in water level either and only was visible as surface fluctuations. However, we noticed mixing during some wave overtopping events, depending on the water level, for higher levels we could not observe.\\

Wind force, on the other hand, caused a big impact in the estuarine dynamics, and was demonstrated to cause changes in density layers during and after wind influence. It is shown that wind stress moved the layers causing upwelling during the wind events and, when stopped, stratified the water column but with different density changes between the surface and the bottom, demonstrating there was mixing present during the wind event. $W$, $\phi$, and $N^2$ show remarkable predictive power through the two study periods. Upwelling magnitude is highly dependent on the duration of winds relative to the baroclinic setup time, including cases where wind duration is several times longer than the setup time. The observed patterns during the wind events are consistent with the theory.\\

Pescadero Estuary is an ideal site for studying wind effects in the stratification due to the bi-directionality of wind caused by the estuarine morphology. That means that the study can be extrapolated easily to other bar-built estuaries. Also, for future studies, three-layer bi-dimensional models can be applied to account for all of the relevant processes, considering the high consistency between wind stress and the estuary.\\

In summary, wind stress significantly contributes to mixing the water column throw upwelling in a small and highly stratified estuary. The density structure changes can result from a variety of processes. Which of these processes are relevant in a specific estuary depends on its morphology, its salinity, and also its stratification. The stratification and other conditions such as the area and depth at the estuary affect the occurrence of upwelling and mixing events. In the same way, changes in the wind stress events' magnitude and duration or in the estuarine morphologic conditions, including changes in water level, will have consequences for stratification and deep-water renewal mainly because of a change in upwelling occurrence. The results of this work could be applied to other small estuaries with seasonal or permanent closures like coastal lagoons.\\

\end{document}