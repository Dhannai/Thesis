\documentclass[tesis.tex]{subfiles}
\begin{document}
  
\section{Conclusion}
- We were able to identify a change in time of the stratification conditions of pescadero, changes that are due to wind, discharge and wave overtopping.

Se estuadiaron los efectos externos sobre el estuario pescadero que causaban cambios en la estratificacion del estuario de los cuales se descurbio que el mas importante es el viento, luego viene el caudal y finalmente no se pudo (o quizas si) cuantificar con exactitud la inflencia de la entrada de las olas pero si se identifico que su efecto es el menos importante de los analizados en este TRABAJO



Futuras investigaciones?
Aplicaciones a otros estuarios?

W, phi, N2, etc shows remarkable predictive power through the three study
events (conclusion??)

Upwelling magnitude is highly dependent on the duration of
winds relative to the baroclinic setup time, including cases
where wind duration is several times longer than the setup
time, and particularly for rotationally influenced systems
(de la Fuente et al. 2008)

The complex current and isotherm patterns observed during relaxation are consistent with theory



\end{document}